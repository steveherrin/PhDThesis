Observations of neutrino flavor oscillations have definitively shown that neutrinos have mass. Since this discovery, much progress has been made at measuring the neutrino mass differences and lepton mixing angles that characterize these oscillations. However, the origin and absolute scale of neutrino masses remain unknown. Discovery of a hypothetical process known as neutrinoless double beta decay would show that neutrinos are Majorana particles. Majorana particles can acquire mass via different mechanisms than do Dirac particles, which could in turn explain why neutrino masses are so much smaller than those of other fermions. Additionally, if neutrinoless double beta decay is observed, the absolute masses of the neutrinos can be deduced from the decay rate.

The Enriched Xenon Observatory (EXO) is a series of experiments searching for the neutrinoless double beta decay of \xenon{136}. The first experiment, EXO-200, began operation in 2011 and makes use of \SI{200}{\kg} of xenon enriched to \SI{80.6}{\percent} in \xenon{136}. Data from EXO-200 has been used to measure the half-life of the standard-model-allowed mode of two-neutrino-emitting double beta decay of \xenon{136}. This data has also been used to set a lower limit on the half life of the neutrinoless decay mode. Following these initial results, much effort has gone into improving the understanding of the EXO-200 detector. The analysis presented here makes use of this improved understanding to more precisely measure the half-life of the two-neutrino-emitting mode and to set limits on the half-lives of exotic, Majoron-emitting modes of neutrinoless double beta decay. Also presented here is a new measurement of the cosmic muon flux at the WIPP underground site where EXO-200 is located. This is important for understanding cosmogenic backgrounds in EXO-200.
