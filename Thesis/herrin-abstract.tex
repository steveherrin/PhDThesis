In the decade since neutrino oscillation experiments definitively proved that neutrinos have mass, much progress has been made at measuring the mass splittings and mixing angles that characterize these oscillations. However, the origin and absolute scale neutrino masses remains known. Discovery of a hypothetical process known as neutrinoless double beta decay could help answer these questions.

The Enriched Xenon Observatory (EXO) is a series of experiments searching for the neutrinoless double beta decay of xenon-136. The first experiment, EXO-200, has been taking data since 2011. It has measured the standard model allowed mode of two neutrino emitting double beta decay and set a limit on the neutrinoless mode with its so-called Run2a data set. Since then, much effort has gone into improving the understanding of the detector. The analysis presented here makes use of this improved understanding to precisely measure the half life of the two neutrino emitting mode and to set limits on exotic, Majoron-emitting modes of neutrinoless double beta decay. It also presents a new measurement of the cosmic muon flux at the WIPP underground site where EXO-200 is located, which is important for understanding cosmogenic backgrounds in EXO-200.
