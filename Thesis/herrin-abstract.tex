Observations of neutrino flavor oscillations definitively demonstrate that neutrinos have mass. Since the discovery of these oscillations, much progress has been made at measuring the neutrino mass differences and lepton mixing angles that characterize them. However, the origin and absolute scale of neutrino masses remain unknown. Unique among fermions, neutrinos can be Majorana particles, which could provide an explanation for neutrino masses. Discovery of a hypothetical process known as neutrinoless double beta decay would show that neutrinos are Majorana particles and determine the mass scale for neutrinos.

The Enriched Xenon Observatory (EXO) is a series of experiments searching for the neutrinoless double beta decay of \xenon{136}. The first experiment, EXO-200, began operation in 2011 and makes use of \SI{200}{\kg} of xenon enriched to \SI{80.6}{\percent} in \xenon{136}. The analysis presented here makes use data from EXO-200 to obtain a precise measurement of the half-life for the two-neutrino-emitting mode and to set limits on the half-lives for exotic, Majoron-emitting modes of neutrinoless double beta decay. Data from EXO-200 is also used to produce a measurement of the cosmic muon flux at the WIPP underground site where EXO-200 is located.
