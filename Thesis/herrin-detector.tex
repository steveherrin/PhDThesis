\documentclass[herrin-thesis.tex]{subfiles}
\begin{document}

\chapter{The EXO-200 Detector}
\label{ch:detector}

\section{Creating a Sensitive Detector}
The number of decays of a radioactive element in a given period of time follow a Poisson distribution, provided the half life is much longer than the observation time. This is clearly the case for a double beta decay experiment. The expected signal in such a case is
\begin{equation}
N_{s} = \epsilon \frac{a M}{A} \frac{t \ln2}{T_{1/2}}
\label{eq:detector_N_expected}
\end{equation}
events, where  \(\epsilon\) is the efficiency for detection, \(a\) is the isotopic abundance, \(M\) is the total mass of the element, \(A\) is the mass of a single atom, \(t\) is the total observation time, and \(T_{1/2}\) is the half life.

If an experiment observes no decays, then, it sets an lower limit on the half life substituting for \(N_s\) a number that corresponds to a desired confidence level for the Poisson distribution. The sensitivity to the half-life, then, goes like
\begin{equation}
S(T_{1/2}) \propto \epsilon t \frac{a M}{A}
\label{eq:detector_bgfree_sensitivity}
\end{equation}

Now suppose that an experiment has some number \(N_b\) of background events in the region of interest. It is reasonable to assume that the rate of background events \(b\) is flat over an energy-based region of interest, the width of which increases with the detector resolution \(\Gamma\). The number of background events should increase with the exposure \(M t\). That is, \(N_b \propto b M t \Gamma\). The confidence limit on \(N_s\) must take into account the Poisson counting uncertainty on \((N_s + N_b)\), which will be dominated by the uncertainty on \(N_b\), and so the sensitivity instead goes like
\begin{equation}
S(T_{1/2}) \propto t \epsilon \frac{a M}{A} \frac{1}{\Delta N_b} \approx t \epsilon \frac{a M}{A} \frac{1}{\sqrt{N_b}} \propto \epsilon \frac{a}{A} \sqrt{\frac{M t}{b \Gamma}}
\label{eq:detector_bg_sensitivity}
\end{equation}
where the approximation that the uncertainty \(\Delta N_b = \sqrt{N_b}\) is valid when \(N_b\) is large enough that Gaussian statistics apply. Poisson statistics on \(N_b\) create a transition between \cref{eq:detector_bgfree_sensitivity,eq:detector_bg_sensitivity}.

The design of a detector for a double beta decay experiment is guided by \cref{eq:detector_bgfree_sensitivity,eq:detector_bg_sensitivity}. It must be able to contain a large mass \(M\) of material highly-enriched in the isotope of interest \((a)\). It must also have a good energy resolution (small \(\Gamma\)) and be constructed to minimize backgrounds (small \(b\)). In a search for \zeronu{}, recalling \cref{eq:nu_zeronu_rate}, an experiment with large backgrounds will decrease the smallest  \(\langle m_{\nu} \rangle\) it is sensitive to as \((t)^{-1/4}\), which is very slowly.

\section{Time Projection Chamber}
The EXO-200 detector uses liquid xenon as both the source and the detector of double beta decays. It takes the form of a Time Projection Chamber (TPC), with the xenon in the liquid phase. A TPC configuration collects both the ionization and scintillation signals so they can be combined to improve energy resolution as discussed in \cref{ch:liquidxe}. A TPC also collects information about event topology, which is useful for background rejection. \Cref{fig:detector_tpc_cartoon} illustrates the TPC concept.
\begin{figure}
\centering
\includegraphics[width=0.6\textwidth]{./figures/detector_tpc_schematic.pdf}
\caption[A conceptual drawing of EXO-200]{The concept for EXO-200. The TPC is divided into two identical halves by a cathode. When energy is deposited in the liquid xenon, it creates both scintillation light and ionization. APDs behind the anode wire planes collect the scintillation signal. Ionization drifts to the anode and is read out on two crossed wire planes.}
\label{fig:detector_tpc_cartoon}
\end{figure}

\subsection{Scintillation Readout}

\begin{figure}
\centering
\includegraphics[width=0.6\textwidth]{./photos/detector_half.jpg}
\caption[Half of the EXO-200 detector]{One half of the detector while being assembled. The APDs are visible in their platter and the flex cables can be seen around the edges. The anode wire planes are above the APD plane and cross each other at \ang{60}. PTFE tiles line the inside of the field-shaping rings. The cathode grid has not been installed in this photo, but would be at the top.}
\label{fig:detector_half_photo}
\end{figure}

The scintillation light is collected by Large Area Avalanche PhotoDiodes (LAAPDS, or APDs). APDs can be manufactured with significantly lower radioactivity than traditional photomultiplier tubes. They operate well at cryogenic temperatures, and have a higher quantum efficiency than PMTs at \SI{178}{\nm}. However, APDs do suffer from increased noise and reduced gain compared to PMTs, but this is less of a concern at cryogenic temperatures, and for the relatively high energies associated with double beta decay.

The 468 APDs are housed two identical platters on either end of the TPC. The APDs are electrically connected to the platters, which provide \about~\SI{1.4}{\kV} bias. The platters are plated with \ce{Al} and \ce{MgF2} for improved reflectivity. PTFE tiles line the walls of the TPC and also provide reflection to improve light collection efficiency. One device is missing from each platter in order to make room for a diffuser that can deliver light from an external laser for calibration purposes.

The APDs are electrically grouped together in gangs of 5 to 7. A phosphor bronze ``spider'' provides a common electrical connection to a gang, and mechanically holds them in the platter. APDs vary significantly in voltage needed to achieve a desired product of quantum efficiency and gain, and so the gangs are chosen in order to match electrically-similar devices\cite{Neilson:2009fk}. A trim voltage is then applied so that all gangs have similar performance.

\subsection{Ionization Readout}

\begin{figure}
\centering
\includegraphics[width=0.6\textwidth]{./figures/detector_wire_geometry.pdf}
\caption[Geometry of the ionization readout wire planes]{The geometry of the ionization readout. An ionization cloud drifts past the \(v\) wires, inducing a signal in them. The cloud is then collected on the \(u\) wires. For simplicity, the wires are shown collinear here. In reality, they are angled \ang{60} from each other.}
\label{fig:detector_wire_geometry}
\end{figure}

A central cathode divides EXO-200 into two drift regions. This cathode creates a drift field that drifts ionization electrons to the anodes, at either end of the TPC. The anodes consist of two crossed wire planes, angled \ang{60} to each other. The plane farther from the cathode, denoted \(u\), is held at virtual ground and collects the ionization. The plane closer to the cathode is denoted \(v\) and is biased to be fully transparent to drifting ionization. The \(v\) wire plane shields the \(u\) wire plane from induction effects. The induced signal on the \(v\) wires also provides a transverse coordinate. This, along with knowledge of which \(u\) wire collected the charge, provides a location for the event in the transverse plane.

Each wire plane consists of 114 wires. These wires were formed as triplets by photoetching sheets of phosphor bronze. This gives 38 electrical readouts per plane. Each wire plane is 95.8\% transparent to scintillation light. The cathode is also formed from two pieces of etched phosphor bronze and is 90\% transparent to light.

The wire planes are spaced \SI{6}{\mm} apart, and the collection plane is \SI{6}{\mm} from the face of the the platter that holds the APDs. The cathode, biased to \SI{-8}{\kV}, is \SI{19.2}{\cm} from the induction wire plane. The cathode and cylindrical field shaping rings along the sides of the drift region create a \SI{374}{\V\per\cm} field in the main drift region. Charge in this region drifts at \SI{1.71}{\mm\per\micro\s}. The field is stronger between the \(u\) and \(v\) wires, and charge drifts about \SI{2.2}{\mm\per\micro\s} in this region.

\subsection{Construction}

\begin{figure}
\centering
\includegraphics[width=0.6\textwidth]{./photos/detector_TPC.pdf}
\caption[Cutaway view of the EXO-200 detector]{A cutaway view of the EXO-200 detector and liquid xenon vessel.}
\label{fig:detector_TPC_vessel}
\end{figure}

EXO-200 is constructed out of\todo{Mention low background copper, describe legs, etc.}

\section{Calibration}

\begin{figure}[htb]
\centering
\begin{subfigure}[c]{0.33\linewidth}
\includegraphics[width=\textwidth]{./photos/source_capsule.png}
\end{subfigure}\hspace{0.05\linewidth}\hfill%
\begin{subfigure}[c]{0.60\linewidth}
\includegraphics[width=\textwidth]{./photos/calibration_tubing_cropped.png}
\end{subfigure}
\caption[The calibration system]{To calibrate the detector, a tiny source capsules (left) containing a radioisotope can be deployed to many positions just outside the detector through a guide tube system (right).}
\label{fig:detector_calibration}
\end{figure}

Routine calibrations help monitor the performance and characterize the energy response of EXO-200. A guide tube system runs just outside the TPC, providing access to both anodes and three positions around the circumference of the cathode. In a typical calibration run, a miniature radioactive source is pushed to one of these positions by a long cable. After enough time passes to accumulate enough calibration data, the source is retracted and normal physics operations resume.

Currently, EXO-200 calibrates with three different isotopes. \isotope{137}{Cs} emits a \SI{662}{\keV} gamma ray when it decays. \isotope{60}{Co} decays emit both \SI{1173}{\keV} and \SI{1333}{\keV} gamma rays simultaneously. \isotope{228}{Th} eventually decays to \isotope{208}{Tl}, which emits a \SI{2615}{\keV} gamma ray when it decays. Typically, short (\SI{2}{hr}) \isotope{228}{Th} calibrations are taken a few times per week, chiefly to measure the electron lifetime (see \cref{ch:electronlifetime}). Several times per year, longer campaigns with the full suite of sources establish the energy response of the detector.


\section{Xenon Handling}

As the name suggests, EXO-200 makes use of \SI{200}{\kg} of xenon enriched to 80.6\% in \xenon{136}. Of the remaining fraction of the xenon, isotope 134 comprises 19.1\%, and lighter natural isotopes are present in trace amounts.

\section{Infrastructure}

\begin{figure}
\centering
\includegraphics[width=1.0\textwidth]{./photos/detector_cleanroom.pdf}
\caption[Cutaway view of EXO-200 as installed]{A cutaway view of EXO-200 as installed at WIPP. The TPC vessel is surrounded by cold HFE. A cryostat insulates the HFE and liquid xenon. Lead surrounds the cryostat to shield from external radiation. All of this is located inside a clean room, which is surrounded with an active muon veto.}
\label{fig:detector_cleanroom}
\end{figure}

\subsection{Cryostat and Clean Rooms}

\subsection{Muon Veto}

\subsection{WIPP}

\begin{figure}[htb]
\centering
\begin{subfigure}[c]{0.30\linewidth}
\includegraphics[width=\textwidth]{./figures/detector_wipp_map.pdf}
\end{subfigure}\hspace{0.05\linewidth}\hfill%
\begin{subfigure}[c]{0.60\linewidth}
\includegraphics[width=\textwidth]{./photos/wipp_site_annotated.png}
\end{subfigure}
\caption[The WIPP Site]{The left shows the location of the WIPP site on a map of New Mexico. The right shows a detailed view of the WIPP site. EXO-200 is located in the North Experimental Area, approximately \SI{655}{\m} underground.}
\label{fig:detector_wipp}
\end{figure}

In order to shield from cosmic rays that are a potential background, EXO-200 is located \(\sim\)~\SI{655}{\m} underground at the Department of Energy's Waste Isolation Pilot Plant (WIPP) in southeastern New Mexico. WIPP is a salt mine, and its primary purpose is the permanent disposal of transuranic waste. The north end of the mine, far from the waste, serves as a suitable site for low-background experiments.

The rock overburden provides \SI{1480}{\hecto\g\per\square\cm} shielding from cosmic rays (see \cref{ch:muons}), and the salt walls are naturally low in radionuclides. Direct counting finds a contamination of \SI[per-mode=symbol]{27\pm2d-9}{\g\per\g} of \isotope{238}{U}, \SI[per-mode=symbol]{66\pm2d-9}{\g\per\g} of \isotope{232}{Th}, and \SI[per-mode=symbol]{124\pm2}{\g\per\g} of \isotope{40}{K}\cite{Auger:2012dq}. Little radon emanates from the walls, such that radon levels are similar to those found at the surface.

\end{document}
