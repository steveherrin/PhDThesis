\documentclass[herrin-thesis.tex]{subfiles}
\begin{document}

\chapter{The EXO-200 Detector}
\label{ch:detector}

\section{Creating a Sensitive Detector}
The number of decays of a radioactive element in a given period of time follow a Poisson distribution, provided the half life is much longer than the observation time. This is clearly the case for a double beta decay experiment. The expected signal in such a case is
\begin{equation}
N_{s} = \epsilon \frac{a M}{A} \frac{t \ln2}{T_{1/2}}
\label{eq:detector_N_expected}
\end{equation}
events, where  \(\epsilon\) is the efficiency for detection, \(a\) is the isotopic abundance, \(M\) is the total mass of the element, \(A\) is the mass of a single atom, \(t\) is the total observation time, and \(T_{1/2}\) is the half life.

If an experiment observes no decays, then, it sets an lower limit on the half life substituting for \(N_s\) a number that corresponds to a desired confidence level for the Poisson distribution. The sensitivity to the half-life, then, goes like
\begin{equation}
S(T_{1/2}) \propto \epsilon t \frac{a M}{A}
\label{eq:detector_bgfree_sensitivity}
\end{equation}

Now suppose that an experiment has some number \(N_b\) of background events in the region of interest. It is reasonable to assume that the rate of background events \(b\) is flat over an energy-based region of interest, the width of which increases with the detector resolution \(\Gamma\). The number of background events should increase with the exposure \(M t\). That is, \(N_b \propto b M t \Gamma\). The confidence limit on \(N_s\) must take into account the Poisson counting uncertainty on \((N_s + N_b)\), which will be dominated by the uncertainty on \(N_b\), and so the sensitivity instead goes like
\begin{equation}
S(T_{1/2}) \propto t \epsilon \frac{a M}{A} \frac{1}{\Delta N_b} \approx t \epsilon \frac{a M}{A} \frac{1}{\sqrt{N_b}} \propto \epsilon \frac{a}{A} \sqrt{\frac{M t}{b \Gamma}}
\label{eq:detector_bg_sensitivity}
\end{equation}
where the approximation that the uncertainty \(\Delta N_b = \sqrt{N_b}\) is valid when \(N_b\) is large enough that Gaussian statistics apply. Poisson statistics on \(N_b\) create a transition between \cref{eq:detector_bgfree_sensitivity,eq:detector_bg_sensitivity}.

The design of a detector for a double beta decay experiment is guided by \cref{eq:detector_bgfree_sensitivity,eq:detector_bg_sensitivity}. It must be able to contain a large mass \(M\) of material highly-enriched in the isotope of interest \((a)\). It must also have a good energy resolution (small \(\Gamma\)) and be constructed to minimize backgrounds (small \(b\)). In a search for \zeronu{}, recalling \cref{eq:nu_zeronu_rate}, an experiment with large backgrounds will decrease the smallest  \(\langle m_{\nu} \rangle\) it is sensitive to as \((t)^{-1/4}\), which is very slowly.

\section{Time Projection Chamber}
The EXO-200 detector uses liquid xenon as both the source and the detector of double beta decays. It takes the form of a Time Projection Chamber (TPC), with the xenon in the liquid phase. A TPC configuration collects both the ionization and scintillation signals so they can be combined to improve energy resolution as discussed in \cref{ch:liquidxe}. A TPC also collects information about event topology, which is useful for background rejection. \Cref{fig:detector_tpc_cartoon} illustrates the TPC concept.
\begin{figure}
\centering
\includegraphics[width=0.6\textwidth]{./figures/detector_tpc_schematic.pdf}
\caption[A schematic of the EXO-200 TPC]{The concept for EXO-200. The TPC is divided into two identical halves by a cathode. When energy is deposited in the liquid xenon, it creates both scintillation light and ionization. Avalanche PhotoDiodes behind the anode wire planes collect the scintillation signal. Ionization drifts to the anode and is read out on two crossed wire planes. The ``u'' collection plane is held at virtual ground and collects the ionization. The wires that collect charge provide one transverse coordinate. The ``v'' induction plane shields the collection wires and determines the other transverse coordinate.}
\label{fig:detector_tpc_cartoon}
\end{figure}

\subsection{Ionization Readout}

\begin{figure}
\centering
\includegraphics[width=0.6\textwidth]{./figures/detector_wire_geometry.pdf}
\caption[Geometry of the ionization readout wire planes]{The geometry of the ionization readout. An ionization cloud drifts past the v wires, inducing a signal in them. The cloud is then collected on the u wires. For simplicity, the wires are shown collinear here. In reality, they are angled \ang{60} from each other.}
\label{fig:detector_wire_geometry}
\end{figure}

\subsection{Scintillation Readout}

\section{Calibration}

\begin{figure}[htb]
\centering
\begin{subfigure}[c]{0.33\linewidth}
\includegraphics[width=\textwidth]{./photos/source_capsule.png}
\end{subfigure}\hspace{0.05\linewidth}\hfill%
\begin{subfigure}[c]{0.60\linewidth}
\includegraphics[width=\textwidth]{./photos/calibration_tubing_cropped.png}
\end{subfigure}
\caption[The Calibration System]{To calibrate the detector, a tiny source capsules (left) containing a radioisotope can be deployed to many positions just outside the detector through a guide tube system (right).}
\label{fig:detector_calibration}
\end{figure}

\section{Xenon Handling}

As the name suggests, EXO-200 makes use of \SI{200}{\kg} of xenon enriched to 80.6\% in \xenon{136}. Of the remaining fraction of the xenon, isotope 134 comprises 19.1\%, and lighter natural isotopes are present in trace amounts. \isotope{85}{Kr} is present at \SI[per-mode=symbol]{25.5\pm3.0d-12}{\g\per\g}\cite{Dobi:2012nx}. This low contamination and low Q value mean it is not problematic for EXO-200.

\section{Shielding}

\subsection{Cryostat and Clean Rooms}

\subsection{WIPP}

\begin{figure}[htb]
\centering
\begin{subfigure}[c]{0.30\linewidth}
\includegraphics[width=\textwidth]{./figures/detector_wipp_map.pdf}
\end{subfigure}\hspace{0.05\linewidth}\hfill%
\begin{subfigure}[c]{0.60\linewidth}
\includegraphics[width=\textwidth]{./photos/wipp_site_annotated.png}
\end{subfigure}
\caption[The WIPP Site]{The left shows the location of the WIPP site on a map of New Mexico. The right shows a detailed view of the WIPP site. EXO-200 is located in the North Experimental Area, approximately \SI{655}{\m} underground.}
\label{fig:detector_wipp}
\end{figure}

In order to shield from cosmic rays that are a potential background, EXO-200 is located \(\sim\)~\SI{655}{\m} underground at the Department of Energy's Waste Isolation Pilot Plant (WIPP) in southeastern New Mexico. WIPP is a salt mine, and its primary purpose is the permanent disposal of transuranic waste. The north end of the mine, far from the waste, serves as a suitable site for low-background experiments.

The rock overburden provides \SI{1480}{\hecto\g\per\square\cm} shielding from cosmic rays (see \cref{ch:muons}), and the salt walls are naturally low in radionuclides. Direct counting finds a contamination of \SI[per-mode=symbol]{27\pm2d-9}{\g\per\g} of \isotope{238}{U}, \SI[per-mode=symbol]{66\pm2d-9}{\g\per\g} of \isotope{232}{Th}, and \SI[per-mode=symbol]{124\pm2}{\g\per\g} of \isotope{40}{K}\todo{Find a source for this besides Russell's thesis.}. Little radon emanates from the walls, such that radon levels are similar to those found at the surface.

\subsection{Muon Veto}


\end{document}
