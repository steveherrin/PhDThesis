\documentclass[herrin-thesis.tex]{subfiles}
\begin{document}

\chapter{The EXO-200 Detector}
\label{ch:detector}

\section{Detector Sensitivity}
The number of decays of a radioactive element in a given period of time follow a Poisson distribution, provided the half life is much longer than the observation time. This is clearly the case for a double beta decay experiment. The expected signal in such a case is
\begin{equation}
N_{s} = \epsilon \frac{a M}{A} \frac{t \ln2}{T_{1/2}}
\label{eq:detector_N_expected}
\end{equation}
events, where  \(\epsilon\) is the efficiency for detection, \(a\) is the isotopic abundance, \(M\) is the total mass of the element, \(A\) is the mass of a single atom, \(t\) is the total observation time, and \(T_{1/2}\) is the half life.

If an experiment observes no decays, then, it sets an lower limit on the half life substituting for \(N_s\) a number that corresponds to a desired confidence level for the Poisson distribution. The sensitivity to the half-life, then, goes like
\begin{equation}
S(T_{1/2}) \propto \epsilon t \frac{a M}{A}
\label{eq:detector_bgfree_sensitivity}
\end{equation}

Now suppose that an experiment has some number \(N_b\) of background events in the region of interest. It is reasonable to assume that the rate of background events \(b\) is flat over an energy-based region of interest, the width of which increases with the detector resolution \(\Gamma\). The number of background events should increase with the exposure \(M t\). That is, \(N_b \propto b M t \Gamma\). The confidence limit on \(N_s\) must take into account the Poisson counting uncertainty on \((N_s + N_b)\), which will be dominated by the uncertainty on \(N_b\), and so the sensitivity instead goes like
\begin{equation}
S(T_{1/2}) \propto t \epsilon \frac{a M}{A} \frac{1}{\Delta N_b} \approx t \epsilon \frac{a M}{A} \frac{1}{\sqrt{N_b}} \propto \epsilon \frac{a}{A} \sqrt{\frac{M t}{b \Gamma}}
\label{eq:detector_bg_sensitivity}
\end{equation}
where the approximation that the uncertainty \(\Delta N_b = \sqrt{N_b}\) is valid when \(N_b\) is large enough that Gaussian statistics apply. Poisson statistics on \(N_b\) create a transition between \cref{eq:detector_bgfree_sensitivity,eq:detector_bg_sensitivity}.

The design of a detector for a double beta decay experiment is guided by \cref{eq:detector_bgfree_sensitivity,eq:detector_bg_sensitivity}. It must be able to contain a large mass \(M\) of material highly-enriched in the isotope of interest \((a)\). It must also have a good energy resolution (small \(\Gamma\)) and be constructed to minimize backgrounds (small \(b\)). In a search for \zeronu{}, recalling \cref{eq:nu_zeronu_rate}, an experiment with large backgrounds will decrease the smallest  \(\langle m_{\nu} \rangle\) it is sensitive to as \((t)^{-1/4}\), which is very slowly.

\section{Overview}



As the name suggests, EXO-200 makes use of \SI{200}{\kg} of xenon, enriched to 80.6\% in \xenon{136} with clean centrifuges at several laboratories in Russia. Of the remaining fraction of the xenon, isotope 134 comprises 19.1\%, and lighter natural elements are present in trace amounts. \isotope{85}{Kr} is present at \SI[per-mode=symbol]{25.5\pm3.0d-12}{\g\per\g}\cite{Dobi:2012nx}. This low contamination and low Q value mean it is not problematic for EXO-200.

\section{Time Projection Chamber}
EXO-200 uses xenon in the liquid phase to make a Time Projection Chamber (TPC). When a process deposits energy in liquid xenon, it creates both scintillation light and ionization. Avalanche PhotoDiodes (APDs) detect the scintillation nearly instantaneously. The ionization, meanwhile, drifts in an applied electric field to crossed wire planes, where it induces a signal in one plane (the ``v'' wires) and is collected on the other plane (the ``u'' wires). \Cref{fig:detector_tpc_cartoon} provides a schematic of the TPC concept for EXO-200. The crossed wire grids provide a 2D projection of the event, while the time between the scintillation signal and the ionization signal provides a \(z\) coordinate. The EXO-200 chamber is actually divided into two separate TPCs by a cathode, biased to \SI{-8.0}{\kV}, producing a drift field of \SI{374}{\V\per\cm}

\begin{figure}
\centering
\includegraphics[width=0.6\textwidth]{./figures/tpc_schematic.pdf}
\caption[A schematic of a TPC]{A schematic of the TPC concept for EXO-200.}
\label{fig:detector_tpc_cartoon}
\end{figure}

The use of a liquid-phase TPC has several advantages for a low-background experiment. The xenon at the outer edges of the detector, outside of the active region, provides modest self shielding against \(\gamma\) rays. The scintillation and the ionization signals can be combined to give good energy resolution\cite{Conti:2003tg}\cite{Aprile:2007hc}. \(\gamma\) ray backgrounds near the Q value of \xenon{136} most often deposit their energy via Compton scattering or electron-positron pair production. These processes create spatially-separated ionization signals, which can then be distinguished from double beta decays, which deposit their energy at a single location in a small volume.

\section{Calibration}

\begin{figure}[htb]
\centering
\begin{subfigure}[c]{0.33\linewidth}
\includegraphics[width=\textwidth]{./photos/source_capsule.png}
\end{subfigure}\hspace{0.05\linewidth}\hfill%
\begin{subfigure}[c]{0.60\linewidth}
\includegraphics[width=\textwidth]{./photos/calibration_tubing_cropped.png}
\end{subfigure}
\caption[The Calibration System]{To calibrate the detector, a tiny source capsules (left) containing a radioisotope can be deployed to many positions just outside the detector through a guide tube system (right).}
\label{fig:detector_calibration}
\end{figure}

\section{WIPP}

\begin{figure}[htb]
\centering
\begin{subfigure}[c]{0.30\linewidth}
\includegraphics[width=\textwidth]{./figures/wipp_map.pdf}
\end{subfigure}\hspace{0.05\linewidth}\hfill%
\begin{subfigure}[c]{0.60\linewidth}
\includegraphics[width=\textwidth]{./photos/wipp_site_annotated.png}
\end{subfigure}
\caption[The WIPP Site]{The left shows the location of the WIPP site on a map of New Mexico. The right shows a detailed view of the WIPP site. EXO-200 is located in the North Experimental Area, approximately \SI{655}{\m} underground.}
\label{fig:detector_wipp}
\end{figure}

In order to shield from cosmic rays that are a potential background, EXO-200 is located \(\sim\)~\SI{655}{\m} underground at the Department of Energy's Waste Isolation Pilot Plant (WIPP) in southeastern New Mexico. WIPP is a salt mine, and its primary purpose is the permanent disposal of transuranic waste. The north end of the mine, far from the waste, serves as a suitable site for low-background experiments.

The rock overburden provides \SI{1480}{\hecto\g\per\square\cm} shielding from cosmic rays (see \cref{ch:muons}), and the salt walls are naturally low in radionuclides. Direct counting finds a contamination of \SI[per-mode=symbol]{27\pm2d-9}{\g\per\g} of \isotope{238}{U}, \SI[per-mode=symbol]{66\pm2d-9}{\g\per\g} of \isotope{232}{Th}, and \SI[per-mode=symbol]{124\pm2}{\g\per\g} of \isotope{40}{K}\todo{Find a source for this besides Russell's thesis.}. Little radon emanates from the walls, such that radon levels are similar to those found at the surface.


\end{document}
