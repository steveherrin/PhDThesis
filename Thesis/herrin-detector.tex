\documentclass[herrin-thesis.tex]{subfiles}
\begin{document}

\chapter{The EXO-200 Detector}
\label{ch:detector}

\section{Overview}


\section{A Xenon Detector}
Xenon has many properties that make it useful for a double beta decay experiment:
\begin{itemize}
\item First and foremost, xenon can serve as both as the source and detector of double beta decay events. This minimizes the other materials needed to build the detector that might be sources of radioactive backgrounds. Electrons from the double beta decay do not have to pass through other media before reaching the detector, allowing fewer energy losses and better energy resolution.
\item The Q value for \xenon{136} decays is is \SI{2457.8}{\keV}\cite{Redshaw:2007cr}, which is higher that most \(\gamma\) rays from common radioactive nuclides. \isotope{208}{Tl}, which occurs on the thorium chain and emits a \SI{2615}{\keV}gamma ray is the notable exception. \(\gamma\) rays with higher energies than the Q value can potentially deposit part of their energy in the detector before scattering out, creating an event with energy close to the Q value.
\item The natural abundance of \xenon{136} is 8.9\%. Furthermore, xenon is a gas at standard temperatures and pressures, making it simple to process and enrich in \xenon{136} using ultracentrifugation.
\item Xenon is a nobel element, and so it is relatively easy to purify of all chemically active contaminants. Furthermore, this purification can be done continuously by recirculating the xenon.
\item The isotopes formed in xenon by cosmogenic activation are short-lived, so the xenon only needs a short period underground and a chemical purification before it is ready to be used.
\item Xenon can be easily reused and transferred between experiments. This allows the opportunity to use xenon in complimentary or novel detector designs. Smaller experiments can help amortize the cost of larger experiments.
\item The barium daughter ion could potentially be tagged, reducing backgrounds immensely. (This technique, however, is not used in EXO-200.)
\end{itemize}

As the name suggests, EXO-200 makes use of \SI{200}{\kg} of xenon, enriched to 80.6\% in \xenon{136} with clean centrifuges at several laboratories in Russia. Of the remaining fraction of the xenon, isotope 134 comprises 19.1\%, and lighter natural elements are present in trace amounts. \isotope{85}{Kr} is present at \SI[per-mode=symbol]{25.5\pm3.0d-12}{\g\per\g}\cite{Dobi:2012nx}. This low contamination and low Q value mean it is not problematic for EXO-200.

\end{document}