\documentclass[herrin-thesis.tex]{subfiles}
\begin{document}
\section{Motivation}
Ideally, EXO-200 would only trigger on events that are due to particles interacting in the liquid xenon. Unlike the detectors at hadron colliders that have a large event rate and must selectively trigger, EXO-200 only sees about one event every 10 seconds. Therefore, the triggers do not need to be sophisticated, and indeed err on the side of including events that may not be interesting.

These simple triggers, however, allow events that are not actually due to interactions in the detector to make it into the data. One common source of spurious events is noise, for example due to microphonic vibration of the signal readout cables due to loud noises. When these noise signals go through processing, they can slow down the processing due to odd signal shapes and large multiplicity. Furthermore, they can masquerade as signals. As described below, one of the most common types of noise can be tagged as a TPC muon. These noise events are frequent enough that the deadtime enforced after a TPC muon would cause a significant hit to live time. And so code was developed to identify and tag noise events.

\section{Types of Noise Events}
\subsection{Unphysically Negative Signals on the \(u\) Wires}
\subsection{``Glitch'' events}
\subsection{APD ``bouncing'' events}

\end{document}