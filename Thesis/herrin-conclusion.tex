\documentclass[herrin-thesis.tex]{subfiles}
\begin{document}

\chapter{Conclusions}
\label{ch:conclusion}

Much work has gone into understanding the EXO-200 detector better. This, in turn, allows a greatly reduced uncertainty on the fiducial volume used for this analysis compared to previous ones. The measurement of the \twonu{} half life of \xenon{136} \(T_{1/2} = (1.982\pm0.015(\text{stat})\pm0.045(\text{sys}))\times10^{21}\si{\year}\) is the most precise measurement of a double beta decay half to date. Low background construction and good background rejection allows EXO-200 top place a competitive limit on the Majoron-emitting decay mode \zeronuXX{} with spectral index 7, \(T_{1/2} > 1.30\times10^{22}\si{\year}\) at 90\% CL, even with roughly half the exposure of the KamLAND-Zen experiment.

Since the Run2a data set, EXO-200 has been running steadily for over a year. The analysis techniques and improved detector understanding obtained in this analysis will be used in conjunction with this increased exposure to yield improved sensitivity to \zeronu{} and the Majoron-emitting \zeronuXpX{} modes.

\end{document}
