\documentclass[herrin-thesis.tex]{subfiles}
\begin{document}

\chapter{Conclusions}
\label{ch:conclusion}

Much work has gone into better understanding the EXO-200 detector. An improved measurement of the vertical muon flux at the WIPP underground site (\(\Phi_v = (4.01\pm0.04\text{(stat)}^{+0.04}_{-0.05}\text{(sys)})\times10^{-7}\text{ Hz}/\text{cm}^2\text{sr}\)) made with EXO-200 data allows better estimates of cosmogenic backgrounds. More importantly, the analysis techniques have been improved and the uncertainties on rate measurements and fiducial volume have been reduced compared to previous analyses. The measurement of the \twonu{} half-life of \xenon{136}: \(T_{1/2} = (2.04\pm0.015(\text{stat})\pm0.075(\text{sys}))\times10^{21}\si{\year}\) is the most precise measurement for this isotope to date. Low-background construction and good background rejection allows EXO-200 to place a competitive limit on the Majoron-emitting decay mode \zeronuXX{} with spectral index 7: \(T_{1/2} > 1.33\times10^{22}\si{\year}\) at 90\% CL, even with roughly half the exposure of the KamLAND-Zen experiment.

Since the end of the Run 2a data set, EXO-200 has been running steadily for over a year and collecting more data. The analysis techniques and improved detector understanding demonstrated in this analysis will be used in conjunction with this increased exposure to yield improved sensitivity to \zeronu{} and the Majoron-emitting \zeronuXpX{} modes in future EXO-200 analyses.

\end{document}
