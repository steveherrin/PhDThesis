\documentclass[herrin-thesis.tex]{subfiles}
\begin{document}

\chapter{Measuring Double-beta Decay}
\label{ch:analysis}

\section{Event Selection}
\subsection{Timing-based Vetoes}
In order to reduce backgrounds (described in \cref{sec:muon_motivation}) due to cosmic-ray muons, two cuts are applied to the data:
\begin{itemize}
\item Events that occur within \SI{0.1}{\s} of a hit in a muon veto system panel are cut.
\item Events that occur within \SI{10}{\s} of a muon passing through the TPC (identified by the method in \cref{ch:muons}) are cut.
\end{itemize}
These cuts are designed to remove as many cosmogenic background events as possible while not cutting a portion of the detector live time so large that the trade-off in signal-to-background ratio is not worth it. The TPC muon cut removes X.X\% of the live time, while the muon veto system panel cut removes X.X\% of the live time. Since these are often coincident, the total impact is X.X\%.\todo{Get the final numbers for these.}

Events occurring near each other in time are likely to be due to a decay of some radioactive contaminant, followed by another decay of a short-lived daughter. For example, the \isotope{222}{Rn} daughter \isotope{214}{Bi} \(\beta^{-}\) decays to \isotope{214}{Po}, which then \(\alpha\) decays with a \SI{164}{\micro\s} half-life.  Events with more than one scintillation signal identified are cut for this reason. Furthermore, any event is vetoed if it occurs within \SI{\pm1}{\s} of another event. The dead-time incurred is X.X\%.

\subsection{Other Cuts}

\subsection{Quantities of Interest}
Once an event passes the data quality cuts

\section{Monte Carlo Simulations of Signals and Backgrounds}

\section{Maximum Likelihood Method}

\section{Measurement of \twonu}

\section{Limits on \(0\nu\beta\beta\chi^0(\chi^0)\)}

\end{document}
