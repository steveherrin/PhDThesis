%\documentclass[12pt]{report}
\documentclass[12pt,twoside]{report}

% note that the document can be single or double sided.  

\usepackage[online]{suthesis-2e}
\usepackage{fixltx2e}
\usepackage{amsmath}
\usepackage{subfiles}
\usepackage{subcaption}
\usepackage[section]{placeins}
\usepackage{siunitx}
\usepackage{todonotes}
\usepackage{nag}
\usepackage[hidelinks]{hyperref}
\usepackage{cleveref}

\newcommand{\addref}{\todo[color=red!40]{Add reference.}}

\newcommand{\isotope}[2]{\(^{#1}\)#2}
\newcommand{\xenon}[1]{\isotope{#1}{Xe}}
\newcommand{\cobalt}[1]{\isotope{#1}{Co}}
\newcommand{\cesium}[1]{\isotope{#1}{Cs}}
\newcommand{\thorium}[1]{\isotope{#1}{Th}}

\newcommand{\zeronu}{\(0\nu\beta\beta\)}
\newcommand{\twonu}{\(2\nu\beta\beta\)}

\reversemarginpar
\setlength{\marginparwidth}{1.25in}

\begin{document}

\title{Something Something\\
         With EXO-200}
\author{Steven Herrin}
\dept{Physics}
\principaladviser{Martin Breidenbach}
\firstreader{Pat Burchat}
\secondreader{Aaron Roodman}
 
\beforepreface

\prefacesection{Abstract}
Observations of neutrino flavor oscillations definitively demonstrate that neutrinos have mass. Since the discovery of these oscillations, much progress has been made at measuring the neutrino mass differences and lepton mixing angles that characterize them. However, the origin and absolute scale of neutrino masses remain unknown. Unique among fermions, neutrinos can be Majorana particles, which provides an explanation for neutrino masses. Discovery of a hypothetical process known as neutrinoless double beta decay would show that neutrinos are Majorana particles and determine the mass scale for neutrinos.

The Enriched Xenon Observatory (EXO) is a series of experiments searching for the neutrinoless double beta decay of \xenon{136}. The first experiment, EXO-200, began operation in 2011 and makes use of \SI{200}{\kg} of xenon enriched to \SI{80.6}{\percent} in \xenon{136}. Data from EXO-200 has been used to measure the half-life of the standard-model-allowed mode of two-neutrino-emitting double beta decay of \xenon{136}. This data has also been used to set a lower limit on the half life of the neutrinoless decay mode. Following these initial results, much effort has gone into improving the understanding of the EXO-200 detector. The analysis presented here makes use of this improved understanding to more precisely measure the half-life of the two-neutrino-emitting mode and to set limits on the half-lives of exotic, Majoron-emitting modes of neutrinoless double beta decay. Also presented here is a new measurement of the cosmic muon flux at the WIPP underground site where EXO-200 is located. This is important for understanding cosmogenic backgrounds in EXO-200.


\prefacesection{Acknowledgements}
Lots of people.

\afterpreface
 
\chapter{Introduction}
\subfile{herrin-introduction}

\chapter{Neutrinos}
\label{ch:neutrinos}
\subfile{herrin-neutrinos}

\chapter{EXO-200}
\label{ch:detector}

\chapter{Electron Lifetime}
\label{ch:electronlifetime}
\subfile{herrin-electronlifetime}

\chapter{Muons}
\label{ch:muons}
\subfile{herrin-muons}

\chapter{Conclusions}
\subfile{herrin-conclusion}

\appendix
\chapter{Tagging Electronic Noise}
\label{app:noisetagger}
\subfile{herrin-appendix-noisetagger}


\bibliographystyle{plain}
\bibliography{herrin-thesis}

\onlinesignature

\end{document}
