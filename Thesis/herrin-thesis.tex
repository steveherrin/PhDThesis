%\documentclass[12pt]{report}
\documentclass[12pt,twoside]{report}

% note that the document can be single or double sided.  

\usepackage[online]{suthesis-2e}
\usepackage{fixltx2e}
\usepackage[sorting=none,firstinits=true]{biblatex}
\usepackage{amsmath}
\usepackage{amssymb}
\usepackage{mathtools}
\usepackage[version=3]{mhchem}
\usepackage{braket}
\usepackage{multirow}
\usepackage{booktabs}
\usepackage{subfiles}
\usepackage{subcaption}
\usepackage{rotating}
\usepackage[section]{placeins}
\usepackage[separate-uncertainty=true]{siunitx}
\usepackage{todonotes}
\usepackage{nag}
\usepackage[hidelinks]{hyperref}
\usepackage{cleveref}

\newcommand{\addref}{\todo[color=red!40]{Add reference.}}

\newcommand{\about}[0]{\(\sim\)}

\newcommand{\isotope}[2]{\ce{^{#1} #2}}
\newcommand{\xenon}[1]{\isotope{#1}{Xe}}
\newcommand{\cobalt}[1]{\isotope{#1}{Co}}
\newcommand{\cesium}[1]{\isotope{#1}{Cs}}
\newcommand{\thorium}[1]{\isotope{#1}{Th}}

\newcommand{\zeronu}{\(0\nu\beta\beta\)}
\newcommand{\zeronuX}{\(0\nu\beta\beta\chi^{0}\)}
\newcommand{\zeronuXpX}{\(0\nu\beta\beta\chi^{0}(\chi^{0})\)}
\newcommand{\zeronuXX}{\(0\nu\beta\beta\chi^{0}\chi^{0}\)}
\newcommand{\twonu}{\(2\nu\beta\beta\)}

\DeclareSIUnit\year{yr}
\DeclareSIUnit\Bq{Bq}


\reversemarginpar
\setlength{\marginparwidth}{1.25in}

\addbibresource{herrin-thesis.bib}

\begin{document}

\title{Double beta decay in xenon-136:\\
	 Measuring the neutrino-emitting mode\\
	 and searching for Majoron-emitting modes}
\author{Steven Herrin}
\dept{Physics}
\principaladviser{Martin Breidenbach}
\firstreader{Patricia Burchat}
\secondreader{Aaron Roodman}
 
\beforepreface

\prefacesection{Abstract}
Observations of neutrino flavor oscillations definitively demonstrate that neutrinos have mass. Since the discovery of these oscillations, much progress has been made at measuring the neutrino mass differences and lepton mixing angles that characterize them. However, the origin and absolute scale of neutrino masses remain unknown. Unique among fermions, neutrinos can be Majorana particles, which provides an explanation for neutrino masses. Discovery of a hypothetical process known as neutrinoless double beta decay would show that neutrinos are Majorana particles and determine the mass scale for neutrinos.

The Enriched Xenon Observatory (EXO) is a series of experiments searching for the neutrinoless double beta decay of \xenon{136}. The first experiment, EXO-200, began operation in 2011 and makes use of \SI{200}{\kg} of xenon enriched to \SI{80.6}{\percent} in \xenon{136}. Data from EXO-200 has been used to measure the half-life of the standard-model-allowed mode of two-neutrino-emitting double beta decay of \xenon{136}. This data has also been used to set a lower limit on the half life of the neutrinoless decay mode. Following these initial results, much effort has gone into improving the understanding of the EXO-200 detector. The analysis presented here makes use of this improved understanding to more precisely measure the half-life of the two-neutrino-emitting mode and to set limits on the half-lives of exotic, Majoron-emitting modes of neutrinoless double beta decay. Also presented here is a new measurement of the cosmic muon flux at the WIPP underground site where EXO-200 is located. This is important for understanding cosmogenic backgrounds in EXO-200.


\prefacesection{Acknowledgements}

I'd like to begin by thanking my advisor, Martin Breidenbach. From working with and watching Marty, I've learned a lot about physics, and even more about building and running a successful physics experiment. I'd also like to thank the rest of my committee, including Patricia Burchat, Giorgio Gratta, Aaron Roodman, and Boris Murmann for their time and feedback as I prepared and defended this dissertation.

My family, including my mother Doyn Kellherhals, my father Stanley Herrin, and my sister Janet Herrin were all so supportive of me throughout graduate school, and I thank them deeply. 

My SLAC labmates, both past and present, have been great to work with. Al Odian, Charles Prescott, and Peter Rowson shared with me their experience and knowledge, and I have learned a lot from them. Working together with Derek Mackay, Ryan MacLellan, and Jesse Wodin to actually build EXO-200 was a great adventure, especially when we finally got everything working.

The EXO collaboration is large, and I don't have enough room to thank everyone individually. Phil Barbeau and Simon Slutsky were both awesome to work with while building EXO-200 and then analyzing some of the first data. Tim Daniels, Michelle Dolinski, and Delia Tosi deserve praise for coordinating EXO-200 runs and being great to work with. So much work has gone into this analysis, and I only begin to scratch the surface by thanking David Auty, Joshua Albert, Sean Daugherty, Clayton Davis, Ralph DeVoe, David Moore, Tessa Johnson, Michael Marino, Igor Ostrovsky, Liang Yang, and Josiah Walton. The Gratta Group (including Giorgio) on campus have been almost like a second lab group, and so I must also thank Francisco LePort, Maria Montero-Diez, Russel Neilson, Kevin O'Sullivan, Karl Twelker for their work and friendship.

I would also like to thank the SLAC staff. Bob Conley deserves special mention for teaching me a lot about plumbing, machining, and welding. I thank Matthias Wittgen for taking over maintenance for the slow controls and freeing up a lot of my time for physics. Ken Fouts, Tony Johnson, J.~J. Russell, Knut Skarpaas VIII, Matt Swift, and Antony Waite have done a lot to get EXO-200 running and keep it going.

The administrative staff at both SLAC and Stanford have been very helpful. Traci Kawakami and Marcia Keating have made my life so much easier. Maria Frank and Judy Meo have also helped me handle a lot of the paperwork that goes with being a graduate student.

The technicians at WIPP work hard to keep EXO-200 from breaking down. Jon Eric Davis and Adam Rivas have been working underground day after day, and I appreciate the years they've spent working on EXO-200.

I am grateful to the United States Department of Energy for their support of EXO-200. I would like to especially thank SLAC National Accelerator Laboratory for being a great place to work over these years. I would also like to thank the Waste Isolation Pilot Plant for their hospitality. Much computing power from the National Energy Research Scientific Computing Center went into this work. The National Science Foundation also supports EXO in the United States. The National Sciences and Engineering Research Council in Canada, the Swiss National Science Foundation, the National Research Foundation of Korea, and the Russian Foundation for Basic Research provide support to EXO worldwide.

\afterpreface
 
\subfile{herrin-introduction}

\subfile{herrin-neutrinos}

\subfile{herrin-liquidxe}

\subfile{herrin-detector}

\subfile{herrin-data}

\subfile{herrin-electronlifetime}

\subfile{herrin-muons}

\subfile{herrin-analysis}

\subfile{herrin-conclusion}

\appendix
\subfile{herrin-appendix-noisetagger}

\subfile{herrin-appendix-lightmap}

\printbibliography

\onlinesignature

\end{document}
