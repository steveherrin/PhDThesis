\documentclass[herrin-thesis.tex]{subfiles}
\begin{document}

\section{Electron Capture on Impurities}
When an electromagnetic process deposits energy in a nobel liquid detector, it ionizes the atoms, producing electrons and ions. Some of the electrons will recombine with the ions, which produces scintillation light. However, if the detector has an applied electric field, then the remaining electrons and ions will drift in opposite directions along the field lines. In a detector consisting of perfectly pure nobel liquid, the electrons would all reach the anode and would be collected for an energy measurement. In a non-ideal detector, however, electronegative impurities can capture the drifting electrons and form ions. The ions are more massive and drift more slowly, and so they escape inclusion in the signal used for energy measurement.

Electronegative impurities may capture electrons in three ways\cite{Aprile:2006fk}. I denote the impurities, which may be atoms or molecules, as \(AB\):
\begin{enumerate}
\item Radiative attachment
\begin{equation}
e^{-} + AB \rightarrow AB^{-} + h \nu
\end{equation}
which has a much smaller cross section than the other processes below.
\item Dissociative attachment
\begin{equation}
\begin{split}
e^{-} + AB \rightarrow e^{-} + AB^{*} \rightarrow A^{+} + B^{-} + e^{-} \\
e^{-} + AB \rightarrow AB^{-} \rightarrow A^{+} + B^{-}
\end{split}
\end{equation}
which requires the electron's energy to be much higher than typically found for an electron drifting in a liquid or dense gas.
\item Three-body attachment through the two-stage Bloch-Bradbury reaction
\begin{equation}
\begin{split}
e^{-} + AB \leftrightarrow (AB^{-})^{*} \\
(AB^{-})^{*} + X \rightarrow AB^{-} + X
\end{split}
\label{eq:3bodyattachment}
\end{equation}
\end{enumerate}

\section{Measuring Electron Lifetime}


\subsection{Method}

\subsection{Comparison to Simulation}

\subsection{Practical Considerations}

Energy resolution limits measurement of long electron lifetimes.

Statistical uncertainties on source peak location.

\section{Measurements of Electron Lifetime in EXO-200}

\subsection{Time Variation and Correction Function}

\subsection{Effects of Electron Lifetime on the Energy Resolution}

\subsection{Electron Lifetime in Low Electric Field}

\subsection{Comparison with Gas Purity Monitor Readings}

\end{document}