\documentclass[herrin-thesis.tex]{subfiles}
\begin{document}

\section{Electron Capture on Impurities}
When an electromagnetic process deposits energy in a nobel liquid detector, it ionizes the atoms, producing electrons and ions. Some of the electrons will recombine with the ions, which produces scintillation light. However, if the detector has an applied electric field, then the remaining electrons and ions will drift in opposite directions along the field lines. In a detector consisting of perfectly pure nobel liquid, the electrons would all reach the anode and would be collected for an energy measurement. In a non-ideal detector, however, electronegative impurities can capture the drifting electrons and form ions. The ions are more massive and drift more slowly, and so they escape inclusion in the signal used for energy measurement.

Electronegative impurities may capture electrons in three ways\cite{Aprile:2006fk}. I denote the impurities, which may be atoms or molecules, as \(AB\):
\begin{enumerate}
\item Radiative attachment
\begin{equation}
e^{-} + AB \rightarrow AB^{-} + h \nu
\end{equation}
which has a much smaller cross section than the other processes below.
\item Dissociative attachment
\begin{equation}
\begin{split}
e^{-} + AB \rightarrow e^{-} + AB^{*} \rightarrow A^{+} + B^{-} + e^{-} \\
e^{-} + AB \rightarrow AB^{-} \rightarrow A^{+} + B^{-}
\end{split}
\end{equation}
which requires the electron's energy to be much higher than typically found for an electron drifting in a liquid or dense gas.
\item Three-body attachment through the two-stage Bloch-Bradbury reaction
\begin{equation}
\begin{split}
e^{-} + AB \leftrightarrow (AB^{-})^{*} \\
(AB^{-})^{*} + X \rightarrow AB^{-} + X
\end{split}
\label{eq:3bodyattachment}
\end{equation}
where X represents the atom or molecule that make up the majority of the liquid.
\end{enumerate}

The three-body reaction shown in \cref{eq:3bodyattachment} releases some amount of energy, given by the \emph{electron affinity} of \(AB\). The electron affinity is positive if \(AB\) is electronegative. Nobel elements have a negative electron affinity, so the reaction does not take place in a pure detector.

The rate of the reaction shown in \cref{eq:3bodyattachment} is given by:
\begin{equation}
\frac{dn_{AB}}{dt} = -k_{3} n_{AB} n_{X} n_{e^{-}}
\label{eq:3bodyreactionrate}
\end{equation}
where \(k_3\) is constant for the 3-body reaction, and \(n_{AB}\), \(n_{X}\), and \(n_{e^{-}}\) are the densities of the impurity, the atoms or molecules of the liquid, and the electrons, respectively. \(k_3\) depends on the species of the impurity, the species of the liquid, and the electric field strength.

According to \cref{eq:3bodyreactionrate}, electrons will be captured, forming \(AB^{-}\) at a rate proportional to the density of electrons. Thus, the number of free electrons will decay exponentially over time according to:
\begin{equation}
N_{e^{-}}(t) = N_0 \exp (-t/\tau_e)
\label{eq:exponentialtaue}
\end{equation}
where \(N_0\) is the original number of electrons, and \(\tau_e\) is the \emph{electron lifetime}.

In general, there can be several different species of electronegative impurity. In that case, they all contribute to the electron lifetime according to:
\begin{equation}
\tau_e^{-1} = \sum_i k_i n_i
\label{eq:tauedefinition}
\end{equation}
where \(n_i\) is the density of an electronegative impurity and \(k_i\) is the cross section for electron capture by that impurity. For most impurities, \(k_i\) depends on the electric field strength.

\section{Measuring Electron Lifetime}


\subsection{Method}

\subsection{Comparison to Simulation}

\subsection{Practical Considerations}

Energy resolution limits measurement of long electron lifetimes.

Statistical uncertainties on source peak location.

\section{Measurements of Electron Lifetime in EXO-200}

\subsection{Time Variation and Correction Function}

\subsection{Effects of Electron Lifetime on the Energy Resolution}

\subsection{Electron Lifetime in Low Electric Field}

\subsection{Comparison with Gas Purity Monitor Readings}

\end{document}