\documentclass[herrin-thesis.tex]{subfiles}
\begin{document}

\chapter{Introduction}

In recent decades, experiments such as Super-Kamiokande \cite{Fukuda:1998zr}, SNO \cite{Ahmad:2001ys}, KamLAND \cite{PhysRevLett.90.021802}, and Daya Bay \cite{PhysRevLett.108.171803} have observed neutrino flavor oscillation. Together, these and other experiments have measured the mixing angles between the three neutrino flavor states, as well as two of the three mass squared differences between the mass states. However, despite this progress, there are still unanswered questions about neutrinos: Are neutrinos Dirac or Majorana particles? Is the hierarchy of neutrino masses similar to other massive fermions, or is it inverted? What is the absolute mass scale for neutrinos, and why is their mass so much smaller than other fermions?

A process known as neutrinoless double beta decay (\zeronu{}) may provide the answers for all of these questions. This process, which has yet to be observed, requires neutrinos to be Majorana fermions. A family of ``see-saw'' mechanisms could explain why Majorana neutrinos have such small masses compared to charged leptons and quarks. Furthermore, the rate of neutrinoless double beta decay depends on the neutrino mass scale and mass hierarchy.

The Enriched Xenon Observatory (EXO) is a series of experiments that seek to observe neutrinoless double beta decay in \xenon{136}. The first experiment, EXO-200, began taking data in 2011. Data from EXO-200's ``Run 1'' was used to make the first observation in \xenon{136} of the standard-model-allowed mode of two-neutrino-emitting double beta decay (\twonu) \cite{Ackerman:2011gz}. Data from EXO-200's ``Run 2a'' was used to place a lower limit on the half-life of the neutrinoless mode \cite{Auger:2012ar}. The upper limit on neutrino masses derived from this half-life limit almost completely rules out a controversial claim of observation of \zeronu{} in \isotope{76}{Ge} \cite{KlapdorKleingrothaus:2006ff}.
%Following these initial results, much work has been done to better understand the EXO-200 detector and improve analysis techniques. These improvements should allow better measurements of \twonu{} and greater sensitivity to \zeronu{}.

Searching for an extremely rare process has unique difficulties. Natural radioactivity can create interactions in the xenon that could be mistaken for neutrinoless double beta decay. EXO-200, which was built to have a low rate of these background events, allows rejection the remaining background events through two chief means. The first is through energy resolution. For the \zeronu{} decay mode, only two electrons are emitted, and they carry the full energy of the decay, known as the ``Q value''. The signals collected by EXO-200 can be used to precisely determine the energy of an event, rejecting background events with energies away from the Q value of \xenon{136}. The remaining rate of background events with energies in the ``region of interest'' near the Q value can be estimated by looking for correlated gamma lines elsewhere in the energy spectrum. Secondly, EXO-200's time projection chamber design admits the use of event topology to reject some backgrounds in the region of interest.

This dissertation describes a new analysis of the Run 2a data from EXO-200. This analysis makes use of improved analysis techniques and a better understanding of the detector in order to reduce systematic errors. \Cref{ch:neutrinos} describes the theoretical motivation for a double beta decay experiment. \Cref{ch:liquidxe} details the motivation behind the choice of liquid xenon for the experiment and the physics of a liquid xenon detector. A detector sensitive to double beta decay must satisfy many criteria. \Cref{ch:detector} describes these criteria and their realization in EXO-200.

The system that EXO-200 uses to collect data is described in \cref{ch:data}. This chapter also describes how data is processed from its raw form into physically meaningful quantities. One of the largest corrections applied in this processing is a correction for the electron lifetime: how long drifting ionization electrons survive in EXO-200. The physics behind that, the measurement of that quantity, and the correction are described in extra detail in \cref{ch:electronlifetime}.

Finally, two analyses are presented. The first, in \cref{ch:muons}, details a new measurement of the muon flux at the underground WIPP site where EXO-200 is located. The primary benefit of identifying and measuring muons is to reduce and constrain backgrounds that might interfere with double beta decay measurements. The flux measurement is also of interest to any other experiment at the WIPP site. Lastly, \Cref{ch:analysis} presents the methods used to obtain a more precise measurement of the two-neutrino-emitting mode than previously reported by EXO-200 and reports this measurement. It also reprts limits on modes of neutrinoless double beta decay that emit Majoron particles (\zeronuXpX{}).

\end{document}
