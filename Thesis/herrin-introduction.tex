\documentclass[herrin-thesis.tex]{subfiles}
\begin{document}

\chapter{Introduction}

In recent decades, numerous experiments have studied neutrino oscillation. Together, these experiments have determined the mixing angles between the three neutrino flavor states, as well as two of the three mass squared splittings between the mass states. However, despite this progress, there are still unanswered questions about neutrinos: Are neutrinos Dirac or Majorana particles? Is the hierarchy of neutrino masses similar to other massive fermions, or is it inverted? What is the absolute mass scale for neutrinos, and why is their mass so much smaller than other fermions?

A process known as neutrinoless double beta decay may provide the answers for all of these questions. This process, which has yet to be observed, requires neutrinos to be Majorana particles. A family of ``see-saw'' mechanisms could give Majorana neutrinos such small masses, and some of these mechanisms predict new particles, known as Majorons. Furthermore, the rate of neutrinoless double beta decay depends on the mass scale and also the mass hierarchy. 

The Enriched Xenon Observatory (EXO) is a series of experiments that seek to observe neutrinoless double beta decay in \xenon{136}. The first experiment, EXO-200 has been taking data since 2011. It was the first experiment to observe the standard model allowed two-neutrino double beta decay (\twonu)\cite{Ackerman:2011gz}, and has set the a stringent limit on the half life of the neutrinoless mode\cite{Auger:2012ar}. The upper mass limit derived from this almost completely rules out a controversial claim of discovery in \isotope{76}{Ge}\cite{KlapdorKleingrothaus:2006ff}. Presently, EXO-200 is continuing to collect data, while attempting to reduce its systematic errors in order to improve the measurement of \twonu and the limit on \zeronu.

Searching for an extremely rare process has unique difficulties. Natural radioactivity can create interactions in the xenon that could be confused for neutrinoless double beta decay. EXO-200, which was build to have a low rate of these background events, attempts to reject the remaining background events through two chief means. The first is through energy resolution. By precisely determining the energy of an event, EXO-200 can reject events close to the Q value of \xenon{136}, and also estimate what background still remains in the region of interest by looking for correlated gamma lines elsewhere in the spectrum. Secondly, EXO-200's time projection chamber design admits the use of event topology to reject backgrounds in the region of interest.

This dissertation describes a new analysis of data from EXO-200, making use of a better understanding of the detector and reduced systematic errors. \Cref{ch:neutrinos} describes the theoretical motivation for a double-beta decay experiment. \Cref{ch:liquidxe} details the motivation behind the choice of liquid xenon for the experiment and the physics of a liquid xenon detector. A detector sensitive to double beta decay must satisfy many criteria. \Cref{ch:detector} describes these criteria and their realization in EXO-200.

The system that EXO-200 uses to collect data is described in \cref{ch:data}. This chapter also describes how data is processed from its raw form into physically meaningful quantities. One of the largest corrections applied in this processing is a correction for the electron lifetime: how long drifting ionization electrons survive in EXO-200. The physics behind that, the measurement of that quantity, and the correction are given extra detail in \cref{ch:electronlifetime}.

Finally, two analyses are presented. The first, in \cref{ch:muons}, details a new measurement of the muon flux at the underground WIPP site where EXO-200 is located. The primary benefit of identifying and measuring muons is to reduce and constrain backgrounds that might interfere with double beta decay measurements. The flux measurement is also of interest to any other experiment at the WIPP site. \Cref{ch:analysis} presents the methods used to obtain a more precise measurement of the two neutrino emitting mode than previously reported by EXO-200. It also presents limits on modes of neutrinoless double beta decay that emit Majoron particles.

\end{document}
