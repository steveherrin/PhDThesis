\documentclass[herrin-thesis.tex]{subfiles}
\begin{document}

In recent decades, numerous experiments such as Daya Bay, KamLAND, and SuperK \todo{Add citations} have looked at neutrino oscillation. Together, these experiments have determined the mixing angles between the three neutrino flavor states, as well as two of the three mass squared splittings between the mass states. However, despite this progress, there are still unanswered questions about neutrinos: Are neutrinos Dirac or Majorana particles? Is the hierarchy of neutrino masses similar to other massive fermions, or is it inverted? What is the absolute mass scale for neutrinos, and why is their mass so much smaller than other fermions?

A process known as neutrinoless double beta decay may provide the answers for all of these questions. This process, which has yet to be observed, requires neutrinos to be Majorana particles. A family of ``see-saw'' mechanisms could give Majorana neutrinos such small masses. Furthermore, the rate of neutrinoless double beta decay depends on the mass scale and also the mass hierarchy.

The Enriched Xenon Observatory (EXO) is a series of experiments that seek to observe neutrinoless double beta decay in \xenon{136}. The first experiment, EXO-200 has been taking data for approximately two years, and continues to collect more. It was the first experiment to observe the standard model allowed two-neutrino double beta decay (\twonu)\cite{Ackerman:2011gz}, and has set the most stringent limit on the half life of the neutrinoless mode\cite{Auger:2012ar}. The upper mass limit derived from this almost completely rules out a controversial claim of discovery in \isotope{76}{Ge}\todo{Find Klapdor-Kleingrothaus paper.}.

Searching for an extremely rare process has unique difficulties. Natural radioactivity can create interactions in the xenon that could be confused for neutrinoless double beta decay. EXO-200, which was build to be low background, attempts to reject the remaining background events through two chief means. The first is through energy resolution. By precisely determining the energy of an event, EXO-200 can reject events close to the Q value of \xenon{136}, and also estimate what background still remains in the region of interest by looking for other gamma lines in likely background chains. Secondly, EXO-200's time projection chamber design admits the use of event topology to reject backgrounds in the region of interest.

This dissertation details a new analysis of data from EXO-200. First is a discussion neutrino physics and the EXO-200 detector in Chapters \ref{ch:neutrinos} and \ref{ch:detector}. After that, it presents corrections to the scintillation and ionization measurements in order to correct for detector effects and improve the energy resolution in Chapters \ref{ch:scintillation} and \ref{ch:ionization}. These signals combine to give an excellent energy resolution as described in \ref{ch:anticorrelation}. Next, it describes background rejection through tagging cosmic ray muon events \ref{ch:muons} and through identifying Compton-scattering gamma rays through event topology \ref{ch:topology}. Finally, Chapters \ref{ch:analysis} and \ref{ch:results} describe the analysis of the data from EXO-200 in order to derive new limits on neutrinoless double beta decay. 
\end{document}